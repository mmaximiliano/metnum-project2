\section{Conclusiones}
%Esta secci´on debe contener las conclusiones generales del trabajo. Se deben mencionar
%las relaciones de la discusi´on sobre las que se tiene certeza, junto con comentarios
%y observaciones generales aplicables a todo el proceso. Mencionar tambi´en posibles
%extensiones a los m´etodos, experimentos que hayan quedado pendientes, etc.

En este trabajo realizamos un análisis del funcionamiento de kNN y PCA para un
caso muy particular (análisis de polaridad de comentarios de películas), pero
de gran interés para poder realizar en el futuro un análisis social o de mercado,
además de por su posible uso en otros tipos de textos.

Con los distintos experimentos que realizamos, pudimos ver que estos
métodos tiene el potencial de predecir sentimientos con una exactitud
de hasta el 75\% aproximadamente, que no es para nada despreciable.

Sin embargo, estos resultados los conseguimos al utilizar kNN \textbf{sin PCA}.
Como vimos en la sección anterior, usar PCA nos presenta un \textit{trade-off}
entre velocidad y exactitud.

Algo interesante visto en la experimentación fue la relación entre las palabras
filtradas por frecuencia y la calidad de los resultados.
En estos, contrario a nuestras expectativas previas, pudimos ver que hay
palabras poco frecuentes que resultan realmente relevantes.
Consideramos que un posible estudio a realizar en el futuro podría venir por
el lado de ver qué palabras concretas son, para tener una mejor comprensión
de lo que sucede, y probar con una mayor variedad de porcentajes para filtrar palabras.

Finalizando este trabajo, concluimos que kNN es un método simple pero efectivo
para clasificar, al menos en el contexto particular de este trabajo, y que
PCA es una buena herramienta para acompañar a kNN cuando necesitamos acelerar
el mismo, y podemos permitirnos una reducción en la exactitud del análisis.
